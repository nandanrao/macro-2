\documentclass[a4paper,12pt]{article}
\usepackage{mathtools,amsfonts,amssymb,amsmath, bm,commath,multicol}
\usepackage{algorithmicx, tkz-graph, algorithm, fancyhdr, pgfplots, fancyvrb}
\usepackage[noend]{algpseudocode}

\pagestyle{fancy}
\fancyhf{}
\rhead{16/1/2018}
\lhead{PS 0}
\rfoot{\thepage}

\begin{document}

\section*{2}
\subsection*{a}
Firstly, note that in the provided formulation our control variable is consumption($c_t$), the assets ($a_t$) is an endogenous state variable, and the interest rate ($R_t$) is an exogenous state variable. Consumption can be written, by using the constraint, as a function of our state variable $a_t$:
%
$$
c_t = a_t - \frac{a_{t+1}}{R_t}
$$
%
To simplify the formulation of the problem, we create a new endogenous state variable that will also be our control variable, called \textit{savings}:
%
$$
s_{t} = \frac{a_{t+1}}{R_{t}}
$$
%
This allows us to express consumption at time t as an expression of savings, interest rate, and future savings:
%
$$
c_t = s_{t-1}R_{t-1} - \frac{s_{t}R_{t}}{R_{t}} = s_{t-1}R_{t-1} - s_{t}
$$
%
With this formulation, our Bellman equation becomes:
%
$$
V(s_t, R_t) = \max_{s_{t+1} = \pi (s_t, R_t)} \ U(s_t, R_t, s_{t+1}) + \beta \mathbb{E}[ V(s_{t+1}, R_{t+1})]
$$
%
where
%
$$
U(s_t, R_t, s_{t+1}) = u(s_tR_t - s_{t+1}) = u(c_{t+1})
$$
%
and our state consists of endogenous \textit{savings} ($s_t$) and exogenous \textit{interest rate} ($R_t$), and our control variable is the next period's endogenous state ($s_{t+1}$).

\subsection*{c}

\paragraph{FOC} First-order conditions are given by maximizing our value function under our control variable (the output of our policy function), which is also the endogenous state in our formulation, $s_{t+1}$. Note: we use $f'_{x}()$ to denote $\frac{df}{dx}$ for ease of notation.
%
$$
U'_{s_{t+1}}(s_t, R_t, s_{t+1}) + \beta \mathbb{E} [ V'_{s_{t+1}}(s_{t+1}, R_{t+1})] = 0
$$
%
\paragraph{Envelope} We evaluate $V'_{s_{t+1}}(s_{t+1}, R_{t+1})$ directly to obtaint the ``forwarded'' envelope conditions:
%
\begin{align*}
V'_{s_{t+1}}(s_{t+1}, R_{t+1}) =& \\
&U'_{s_{t+1}}(s_{t+1}, R_{t+1}, s_{t+2}) + \\
&\underbrace{U'_{s_{t+2}}(s_{t+1}, R_{t+1}, s_{t+1})\frac{ds_{t+2}}{ds_{t+1}} + \beta \mathbb{E} [ V'_{s_{t+2}}(s_{t+2}, R_{t+1})\frac{ds_{t+2}}{ds_{t+1}}]}_{= 0 \text{ by FOC}}
\end{align*}
%
\paragraph{Euler}
With the forwarded envelope equation, we plug it into the first-order conditions to get our Euler equation:
%
$$
- U'_{s_{t+1}}(s_t, R_t, s_{t+1}) = \beta \mathbb{E} [U'_{s_{t+1}}(s_{t+1}, R_{t+1}, s_{t+2})]
$$
%
\paragraph{Tranversality} A ``terminal'' condition to ensure a solution to our equation is required, we can get this by ensuring a corner solution (\textit{savings}, and therefore \textit{assets} at the end of time are zero), or an interior solution (our utility function ``saturates'', which is to say, flattens with respect to our \textit{savings}):
%
$$
\lim_{t \rightarrow \infty} \ \beta^t\  \mathbb{E}[ U'_{s_t}(s_t, R, s_{t+1})] \ s_t = 0
$$
%


\end{document}